\documentclass{bmvc2k}

%% Enter your paper number here for the review copy
% \bmvcreviewcopy{??}

\title{CVGlobal}

% Enter the paper's authors in order
\addauthor{Leonardo Ruso}{leonardo.russo@inria.fr}{1}
\addauthor{Diego Marcos}{diego.marcos@inria.fr}{1}

% Enter the institutions
% \addinstitution{Name\\Address}
\addinstitution{
 INRIA\\
 Evergreen Team\\
 Montpellier, France
}

\runninghead{Russo, Marcos}{CVGlobal}

% Any macro definitions you would like to include
% These are not defined in the style file, because they don't begin
% with \bmva, so they might conflict with the user's own macros.
% The \bmvaOneDot macro adds a full stop unless there is one in the
% text already.
\def\eg{\emph{e.g}\bmvaOneDot}
\def\Eg{\emph{E.g}\bmvaOneDot}
\def\etal{\emph{et al}\bmvaOneDot}

%------------------------------------------------------------------------- 
% Document starts here
\begin{document}

\maketitle

\begin{abstract}
This document demonstrates the format requirements for papers submitted
to the British Machine Vision Conference.  The format is designed for
easy on-screen reading, and to print well at one or two pages per sheet.
Additional features include: pop-up annotations for
citations~\cite{Authors06,Mermin89}; a margin ruler for reviewing; and a
greatly simplified way of entering multiple authors and institutions.

{\bf All authors are encouraged to read this document}, even if you have
written many papers before.  As well as a description of the format, the
document contains many instructions relating to formatting problems and
errors that are common even in the work of authors who {\em have}
written many papers before.
\end{abstract}

%------------------------------------------------------------------------- 
...

\section{Related Works}
\label{sec:related_works}

Cross-view geo-localization addresses the fundamental challenge of matching images captured from drastically different viewpoints of the same geographic location. This task has evolved from traditional handcrafted feature-based approaches to sophisticated deep learning methods, driven by applications in robotics, augmented reality, and autonomous navigation.

\subsection{Cross-View Geo-localization and Orientation Estimation}

Early pioneering works by Workman and Jacobs~\cite{workman2015predicting,workman2015learning} established the foundation for cross-view matching by demonstrating the potential of CNNs for learning robust feature representations across viewpoint variations. The introduction of benchmark datasets like CVUSA~\cite{workman2015predicting} and later VIGOR~\cite{zhu2021vigor} catalyzed systematic research in this domain, with the latter providing more challenging same-area and cross-area evaluation protocols that better reflect real-world deployment scenarios.

Subsequent developments focused on addressing the inherent domain gap between aerial and ground imagery. Notable approaches include CVM-Net~\cite{hu2018cvm}, which introduced dual-stream CNN architectures with polar transformations to align spatial layouts, and SAFA~\cite{NEURIPS2019_ba2f0015}, which employed attention mechanisms for spatial-aware feature aggregation. Methods like CVFT~\cite{shi2019optimalfeaturetransportcrossview} tackled the domain gap through feature transport modules, while others explored generative approaches to synthesize cross-view correspondences~\cite{regmi2018cross}.

The recognition that orientation estimation is crucial for disambiguation and practical applications led to joint location and orientation frameworks. Recent works have emphasized the importance of spatial awareness in feature representations~\cite{shi2020ilookingatjoint}, particularly for applications requiring precise alignment such as outdoor augmented reality. However, many existing methods either treat orientation as a byproduct of location retrieval or require extensive supervised training with orientation labels.

\subsection{Vision Transformers and Self-Supervised Learning}

The advent of Vision Transformers (ViTs)~\cite{dosovitskiy2020image} has revolutionized cross-view geo-localization by enabling models to capture global spatial relationships through self-attention mechanisms. Transformer-based approaches like L2LTR~\cite{yang2021cross} and TransGeo~\cite{wang2021multi} have demonstrated superior performance over CNN-based methods, leveraging learnable position encodings and global context modeling.

Concurrently, self-supervised learning has emerged as a powerful paradigm for learning robust visual representations without manual annotations. Methods such as MoCo~\cite{he2020momentum}, SimCLR~\cite{chen2020simple}, and BYOL~\cite{grill2020bootstrap} have shown that self-supervised features can match or exceed supervised counterparts on various downstream tasks. DINOv2~\cite{oquab2023dinov2} represents the current state-of-the-art, providing features particularly well-suited for dense prediction tasks and fine-grained visual understanding.

Despite these advances, most transformer-based cross-view methods still rely heavily on supervised learning with orientation labels, requiring extensive annotation efforts. Recent works have begun exploring self-supervised features for geo-localization tasks, but typically treat cross-view matching as standard retrieval without considering the specific geometric constraints and orientation relationships inherent in cross-view scenarios.

\subsection{Multi-Modal Integration and Dataset Limitations}

Recent research has recognized the value of incorporating complementary modalities to improve cross-view matching performance. Sky segmentation techniques help eliminate uninformative regions in ground-level images, focusing attention on building structures and terrain features visible in aerial views~\cite{workman2015predicting}. Advanced monocular depth estimation models like Depth-Anything~\cite{yang2024depth} provide robust geometric cues that enable multi-scale feature aggregation and informed spatial reasoning.

Attention mechanisms have proven particularly effective for cross-view alignment, with cross-attention layers learning to focus on corresponding regions between aerial and ground views. However, most attention-based methods require supervised training with orientation labels and struggle with the limited field-of-view constraints common in real-world applications.

Current benchmark datasets, while valuable, present limitations for comprehensive evaluation. CVUSA focuses primarily on the United States, while VIGOR, though more diverse, still covers a limited geographical scope. These datasets primarily support geo-localization and retrieval tasks, with growing interest in orientation estimation as a distinct but related problem. The lack of large-scale, geographically diverse datasets with comprehensive global coverage has hindered the development of truly robust cross-view methods that generalize across different environmental conditions and cultural contexts.

Our work addresses these limitations by introducing CVGlobal, a large-scale dataset with balanced global representation, and proposing novel orientation-aware methods that combine self-supervised features with multi-modal cues for robust cross-view orientation estimation without requiring extensive supervision.

In this section, we present our methodology for constructing CVGlobal, a large-scale multi-modal dataset that pairs satellite and street-view imagery across diverse global regions. Our approach systematically samples locations from five continents while ensuring geographical diversity and balanced representation between urban and rural environments.

\subsection{Dataset Design and Sampling Strategy}

Our dataset construction methodology is guided by three key principles: \textit{geographical diversity}, \textit{balanced representation}, and \textit{multi-modal consistency}. We define sampling regions across five major continents (North America, Europe, Asia, South America, and Africa), with each continent contributing equally to the final dataset to prevent geographical bias.

For each continent, we establish two distinct sampling regions:
\begin{itemize}
    \item \textbf{Urban regions}: Areas with high population density and significant urban infrastructure
    \item \textbf{Rural regions}: Areas with low population density and predominantly natural or agricultural landscapes
\end{itemize}

The sampling regions are carefully selected to represent diverse climatic, cultural, and developmental contexts within each continent. Table~\ref{tab:sampling_regions} details the specific geographical boundaries for each region.

\begin{table}[t]
\centering
\caption{Geographical sampling regions defined for each continent and environment type.}
\label{tab:sampling_regions}
\begin{tabular}{lllcc}
\toprule
\textbf{Continent} & \textbf{Type} & \textbf{Location} & \textbf{Lat Range} & \textbf{Lon Range} \\
\midrule
North America & Urban & New York City & 40.71°--40.81°N & 74.01°--73.91°W \\
              & Rural & California Farmland & 36.78°--36.88°N & 119.42°--119.32°W \\
\midrule
Europe & Urban & Paris & 48.86°--48.96°N & 2.35°--2.45°E \\
       & Rural & French Countryside & 46.23°--46.33°N & 2.21°--2.31°E \\
\midrule
Asia & Urban & Tokyo & 35.69°--35.79°N & 139.69°--139.79°E \\
     & Rural & Rural India (Agra) & 27.18°--27.28°N & 78.04°--78.14°E \\
\midrule
South America & Urban & São Paulo & 23.55°--23.45°S & 46.63°--46.53°W \\
              & Rural & Brazilian Rainforest & 14.24°--14.13°S & 51.93°--51.83°W \\
\midrule
Africa & Urban & Nairobi & 1.29°--1.19°S & 36.82°--36.92°E \\
       & Rural & Kenyan Savanna & 2.15°--2.05°S & 37.31°--37.41°E \\
\bottomrule
\end{tabular}
\end{table}

\subsection{Urban-Rural Classification}

To ensure accurate labeling of locations as urban or rural, we employ the Global Urban Areas dataset~\cite{schneider2010new}, which provides comprehensive polygon boundaries for urban areas worldwide. For each randomly generated coordinate, we perform a spatial intersection test to determine its classification:

\begin{equation}
\text{Urban}(p) = \begin{cases}
1 & \text{if } p \in \bigcup_{i} U_i \\
0 & \text{otherwise}
\end{cases}
\end{equation}

where $p$ represents a coordinate point and $U_i$ denotes the $i$-th urban area polygon from the Global Urban Areas dataset. This automated classification ensures consistent and objective urban-rural labeling across all geographical regions.

\subsection{Multi-Modal Data Acquisition}

Our data acquisition pipeline consists of three main components: \textit{coordinate generation}, \textit{outdoor location validation}, and \textit{multi-modal image retrieval}.

\subsubsection{Coordinate Generation and Validation}

For each sampling region, we generate random coordinates within the specified geographical boundaries using uniform sampling. Each coordinate undergoes a validation process to ensure data quality:

\begin{enumerate}
    \item \textbf{Urban-Rural Consistency}: Verify that the generated coordinate matches the intended environment type (urban/rural) using the spatial intersection described above.
    \item \textbf{Street View Availability}: Query the Google Street View Metadata API to confirm image availability at the location.
    \item \textbf{Outdoor Location Filtering}: Apply our outdoor detection algorithm to exclude indoor environments.
\end{enumerate}

\subsubsection{Outdoor Detection Algorithm}

To ensure our dataset captures genuine outdoor environments, we implement a robust filtering mechanism that leverages Google Places API data. Our algorithm evaluates each location using the following criteria:

\begin{algorithm}[t]
\caption{Outdoor Location Detection}
\label{alg:outdoor_detection}
\begin{algorithmic}[1]
\Require Street View metadata $M$, Google Maps client $G$
\Ensure Boolean indicating outdoor location
\If{$M.\mathrm{status} \neq \mathrm{OK}$}
    \State \Return False
\EndIf
\If{$M.\mathrm{place\_id}$ is undefined}
    \State \Return True \Comment{Assume outdoor for street-level locations}
\EndIf
\State $\mathrm{details} \gets G.\mathrm{place}(M.\mathrm{place\_id})$
\State $\mathrm{types} \gets \mathrm{details.result.types}$
\State $\mathrm{indoor\_types} \gets \{\mathrm{shopping\_mall}, \mathrm{store}, \mathrm{restaurant}, \mathrm{hospital}, \ldots\}$
\If{$\mathrm{types} \cap \mathrm{indoor\_types} \neq \emptyset$}
    \State \Return False
\Else
    \State \Return True
\EndIf
\end{algorithmic}
\end{algorithm}

This approach is more nuanced than simple keyword filtering, as it distinguishes between genuinely indoor locations (e.g., shopping malls, restaurants) and outdoor points of interest (e.g., parks, monuments) that may also carry establishment tags.

\subsubsection{Image Acquisition and Processing}

For each validated coordinate, we acquire two types of imagery:

\paragraph{Satellite Imagery} We retrieve high-resolution satellite images using the Google Static Maps API with the following specifications:
\begin{itemize}
    \item Resolution: 640x640 pixels
    \item Zoom level: 18 (approximately 1.19 meters/pixel)
    \item Map type: Satellite view
    \item Format: JPEG
\end{itemize}

\paragraph{Street View Imagery} We collect street-view images from four cardinal directions (0°, 90°, 180°, 270°) to provide comprehensive ground-level perspective. Each image has:
\begin{itemize}
    \item Resolution: 640x640 pixels
    \item Field of view: Default Google Street View settings
    \item Format: JPEG
\end{itemize}

The four directional images are horizontally concatenated to create a panoramic representation, resulting in a 2560x640 pixel stitched image that captures the complete ground-level environment.

\subsection{Quality Assurance and Error Handling}

Our data acquisition pipeline implements robust error handling and quality assurance mechanisms:

\subsubsection{Network Resilience}
We employ an exponential backoff retry strategy for API requests, with up to 3 retry attempts for failed connections. This approach handles temporary network issues and API rate limiting gracefully.

\subsubsection{Coordinate Correction}
The Google Street View API may return imagery from coordinates slightly different from the requested location due to road network constraints. We handle this by:
\begin{enumerate}
    \item Recording both original and corrected coordinates
    \item Using corrected coordinates for file naming and deduplication
    \item Ensuring consistent satellite-street view pairing
\end{enumerate}

\subsubsection{Resume Capability}
Our pipeline supports interruption and resumption, checking for existing complete image sets before processing each location. A complete set consists of:
\begin{itemize}
    \item One satellite image
    \item Four directional street view images (0°, 90°, 180°, 270°)
    \item One stitched panoramic image
\end{itemize}

\subsection{Dataset Statistics and Validation}

Throughout the data collection process, we maintain comprehensive statistics including:
\begin{itemize}
    \item Success and failure rates per continent and environment type
    \item API call counts and timing information
    \item Error categorization (metadata failures, download failures, indoor rejections)
    \item Coordinate generation efficiency metrics
\end{itemize}

These statistics are automatically compiled into detailed reports (both human-readable and machine-readable formats) that facilitate dataset validation and quality assessment.

The resulting CVGlobal dataset provides a balanced, geographically diverse collection of paired satellite and street-view imagery suitable for training and evaluating computer vision models across varied global contexts.

\section{Crossview Method}
\label{sec:method}

We propose CroDINO, a novel approach for cross-view orientation estimation that leverages backbone-agnostic feature extraction with orientation-aware token aggregation strategies. Our method addresses the fundamental challenge of aligning ground-level panoramic images with aerial satellite views by exploiting both spatial structure and depth information through a flexible architecture that supports multiple vision foundation models.

\subsection{Problem Formulation}

Given a ground-level panoramic image $I_g$ and an aerial satellite image $I_a$ of the same geographic location, our goal is to estimate the relative orientation $\theta$ between the two views. The ground image is extracted from a 360° panorama using a field-of-view (FOV) window defined by parameters $(f_x, f_y, \psi, \phi)$, where $f_x$ and $f_y$ represent the horizontal and vertical FOV angles, $\psi$ is the yaw (rotation around the vertical axis), and $\phi$ is the pitch (elevation angle).

\subsection{Architecture Overview}

\subsubsection{Backbone-Agnostic Feature Extraction}

CroDINO employs a flexible architecture that can utilize different pre-trained vision models as feature extractors. Our implementation supports multiple backbone architectures including Vision Transformers (DINOv2, CLIP) and Convolutional Neural Networks (ResNet50), allowing for comparative analysis and optimal performance selection.

The backbone-agnostic design consists of:
\begin{itemize}
    \item \textbf{Flexible Feature Extractor}: Support for DINOv2-ViT-B/14, CLIP-ViT-Base-Patch16, or ResNet50 as frozen feature extractors.
    \item \textbf{Unified Token Interface}: Standardized token representation regardless of backbone architecture.
    \item \textbf{Dynamic Grid Adaptation}: Automatic grid size calculation based on token dimensions from different architectures.
\end{itemize}

\subsubsection{Multi-Backbone Token Processing}

The feature extraction process varies by backbone but produces consistent token representations:

\textbf{Vision Transformer Backbones}: For DINOv2 and CLIP models, patch embeddings are extracted and normalized:
\begin{align}
\mathbf{F}_g^{raw} &= \text{Backbone}(I_g) \\
\mathbf{F}_a^{raw} &= \text{Backbone}(I_a) \\
\mathbf{F}_g &= \text{L2Normalize}(\mathbf{F}_g^{raw}[:, 1:, :]) \\
\mathbf{F}_a &= \text{L2Normalize}(\mathbf{F}_a^{raw}[:, 1:, :])
\end{align}

\textbf{CNN Backbones}: For ResNet50, convolutional features are flattened into token-like representations:
\begin{align}
\mathbf{F}_g^{conv} &= \text{ResNet50}(I_g) \\
\mathbf{F}_a^{conv} &= \text{ResNet50}(I_a) \\
\mathbf{F}_g &= \text{Reshape}(\mathbf{F}_g^{conv}, (-1, D)) \\
\mathbf{F}_a &= \text{Reshape}(\mathbf{F}_a^{conv}, (-1, D))
\end{align}

where $D$ represents the feature dimension (768 for ViTs, 2048 for ResNet50). The grid size $G$ is dynamically calculated as $G = \sqrt{N}$ where $N$ is the number of tokens.

\subsection{Orientation-Aware Token Aggregation}

\subsubsection{Sky Filtering and Depth Estimation}

To improve orientation estimation, we incorporate semantic and geometric priors:

\textbf{Sky Segmentation}: We employ a lightweight CNN-based sky filter to identify and mask sky regions in ground images. The sky mask $M_{sky}$ is computed at the patch level using majority voting within each grid cell, producing a binary mask $M_{grid} \in \{0,1\}^{G \times G}$ where 1 indicates ground and 0 indicates sky.

\textbf{Depth Estimation}: We utilize the Depth-Anything model to generate depth maps $D$ for ground images. The depth information is downsampled to match the token grid, providing normalized depth values $d_{i,j} \in [0,1]$ for each spatial location $(i,j)$.

\subsubsection{Multi-Layer Depth-Weighted Token Aggregation}

We introduce a novel aggregation strategy that separates tokens into three depth layers: foreground, middleground, and background. This approach captures the multi-scale nature of visual features in cross-view matching.

\textbf{Vertical Column Analysis}: For each vertical column $j$ in the ground image feature grid, we compute depth-weighted averages over valid (non-sky) tokens:

\begin{align}
\mathbf{t}_j^{fore} &= \frac{\sum_{i: M_{grid}(i,j)=1} w_i^{fore} \cdot \mathbf{f}_{i,j}^g}{\sum_{i: M_{grid}(i,j)=1} w_i^{fore}} \\
\mathbf{t}_j^{mid} &= \frac{\sum_{i: M_{grid}(i,j)=1} w_i^{mid} \cdot \mathbf{f}_{i,j}^g}{\sum_{i: M_{grid}(i,j)=1} w_i^{mid}} \\
\mathbf{t}_j^{back} &= \frac{\sum_{i: M_{grid}(i,j)=1} w_i^{back} \cdot \mathbf{f}_{i,j}^g}{\sum_{i: M_{grid}(i,j)=1} w_i^{back}}
\end{align}

where the depth-dependent weights are defined as:
\begin{align}
w_i^{fore} &= d_{i,j} \\
w_i^{mid} &= \begin{cases} 
\frac{d_{i,j}}{\tau} & \text{if } d_{i,j} \leq 0.5 \\
\frac{1-d_{i,j}}{d_{i,j}} & \text{otherwise}
\end{cases} \\
w_i^{back} &= 1 - d_{i,j}
\end{align}

with threshold $\tau = 0.5$. This weighting scheme emphasizes close objects for foreground, balanced weights for middleground, and distant objects for background layers.

\textbf{Radial Direction Analysis}: For aerial images, we extract features along radial directions from the center, using linear weight progressions:

\begin{align}
\mathbf{r}_\beta^{fore} &= \frac{\sum_{r=0}^{R} w_r^{fore} \cdot \mathbf{f}_{\beta,r}^a}{\sum_{r=0}^{R} w_r^{fore}} \\
\mathbf{r}_\beta^{mid} &= \frac{\sum_{r=0}^{R} w_r^{mid} \cdot \mathbf{f}_{\beta,r}^a}{\sum_{r=0}^{R} w_r^{mid}} \\
\mathbf{r}_\beta^{back} &= \frac{\sum_{r=0}^{R} w_r^{back} \cdot \mathbf{f}_{\beta,r}^a}{\sum_{r=0}^{R} w_r^{back}}
\end{align}

where $\beta$ represents the angular direction, $r$ is the radial distance from center, and the weights follow: $w_r^{fore} = 1-r/R$ (decreasing), $w_r^{back} = r/R$ (increasing), and $w_r^{mid}$ follows a triangular pattern peaking at the center.

\subsection{Orientation Estimation}

\subsubsection{Cross-Modal Alignment via Cosine Distance Minimization}

We estimate orientation by finding the angular offset that minimizes the cosine distance between corresponding vertical and radial feature aggregations. For each candidate orientation $\theta$, we compute the alignment cost:

\begin{align}
\mathcal{L}(\theta) = \frac{1}{G} \sum_{i=0}^{G} \left\| 1 - \begin{bmatrix} \mathbf{t}_{G-1-i}^{fore} \\ \mathbf{t}_{G-1-i}^{mid} \\ \mathbf{t}_{G-1-i}^{back} \end{bmatrix}^T \begin{bmatrix} \mathbf{r}_{\phi(\theta,i)}^{fore} \\ \mathbf{r}_{\phi(\theta,i)}^{mid} \\ \mathbf{r}_{\phi(\theta,i)}^{back} \end{bmatrix} \right\|
\end{align}

where $\phi(\theta,i) = (\lfloor\theta/\Delta\theta\rfloor + i - G/2) \bmod |\mathcal{R}|$ maps vertical columns to radial directions, $\Delta\theta = \text{FOV}_x / G$ is the angular step size, and $G$ is the dynamically calculated grid size.

The optimal orientation is found through exhaustive search:
\begin{align}
\theta^* = \arg\min_{\theta \in [0, 360)} \mathcal{L}(\theta)
\end{align}

\subsubsection{Confidence Estimation}

To assess the reliability of orientation estimates, we compute a confidence score based on the Z-score of the minimum distance:

\begin{align}
\text{confidence} = \frac{\mu(\mathcal{L}) - \min(\mathcal{L})}{\sigma(\mathcal{L})}
\end{align}

where $\mu(\mathcal{L})$ and $\sigma(\mathcal{L})$ are the mean and standard deviation of the loss values across all candidate orientations. Higher confidence scores indicate more reliable orientation estimates.

\subsection{Implementation Details}

\subsubsection{Backbone-Specific Configurations}

Our implementation supports three backbone architectures:

\textbf{DINOv2}: Uses the pre-trained DINOv2-ViT-B/14 model with 14×14 pixel patches, producing a $16 \times 16$ token grid with 768-dimensional features. Positional embeddings are interpolated for different input sizes.

\textbf{CLIP}: Employs CLIP-ViT-Base-Patch16 with 16×16 pixel patches, generating a $14 \times 14$ token grid with 768-dimensional features. L2 normalization is applied to token representations for improved stability.

\textbf{ResNet50}: Uses convolutional features from the penultimate layer, which are reshaped into 2048-dimensional tokens. The spatial resolution depends on the input size and stride configuration.

\subsubsection{Training Strategy and Preprocessing}

Our approach operates in a largely unsupervised manner, leveraging pre-trained features without requiring orientation labels during training. The orientation estimation is performed through geometric alignment of feature aggregations.

\textbf{Data Preprocessing}: We extract random FOV windows from panoramic images with:
\begin{itemize}
    \item Horizontal FOV: $90^\circ$ (configurable)
    \item Vertical FOV: $180^\circ$ 
    \item Random yaw: $\psi \sim \text{Uniform}(0^\circ, 360^\circ)$
    \item Fixed pitch: $\phi = 90^\circ$
\end{itemize}

Aerial images undergo center cropping and resizing to match the ground image dimensions.

\subsubsection{Pipeline Architecture}

The complete pipeline processes image pairs through the following stages:
\begin{enumerate}
    \item \textbf{Feature Extraction}: Backbone-specific token generation with dynamic grid size calculation
    \item \textbf{Sky Segmentation}: CNN-based sky filtering with guided filter refinement
    \item \textbf{Depth Estimation}: Depth-Anything model for geometric understanding
    \item \textbf{Token Aggregation}: Multi-layer depth-weighted aggregation for both vertical and radial directions
    \item \textbf{Orientation Search}: Exhaustive search over discretized orientation space with cosine similarity
\end{enumerate}

All models are implemented in PyTorch and support both CPU and GPU execution. The orientation search space is discretized with angular steps of $\Delta\theta = \text{FOV}_x / G$, where the grid size $G$ is dynamically determined from the backbone's token dimensions. This flexible approach ensures optimal resolution regardless of the chosen backbone architecture.

\bibliography{egbib}
\end{document}
