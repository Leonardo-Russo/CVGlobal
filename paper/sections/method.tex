\section{Crossview Method}
\label{sec:method}

We propose CroDINO, a novel approach for cross-view orientation estimation that leverages the robust feature representations of DINOv2 with orientation-aware token aggregation strategies. Our method addresses the fundamental challenge of aligning ground-level panoramic images with aerial satellite views by exploiting both spatial structure and depth information.

\subsection{Problem Formulation}

Given a ground-level panoramic image $I_g$ and an aerial satellite image $I_a$ of the same geographic location, our goal is to estimate the relative orientation $\theta$ between the two views. The ground image is extracted from a 360° panorama using a field-of-view (FOV) window defined by parameters $(f_x, f_y, \psi, \phi)$, where $f_x$ and $f_y$ represent the horizontal and vertical FOV angles, $\psi$ is the yaw (rotation around the vertical axis), and $\phi$ is the pitch (elevation angle).

\subsection{Architecture Overview}

\subsubsection{Dual-Stream Feature Extraction}

CroDINO builds upon the DINOv2 Vision Transformer architecture, which provides rich, self-supervised feature representations. We modify the standard DINOv2 model to process both ground and aerial images simultaneously while maintaining separate positional embeddings and class tokens for each modality.

The model consists of:
\begin{itemize}
    \item \textbf{Shared Backbone}: We utilize the pre-trained DINOv2-ViT-B/14 as our feature extractor, freezing its parameters to preserve the learned representations.
    \item \textbf{Dual Positional Embeddings}: Separate positional embeddings $\mathbf{E}_{pos}^g$ and $\mathbf{E}_{pos}^a$ for ground and aerial images to account for their different spatial characteristics.
    \item \textbf{Cross-Modal Attention}: A final single-head attention layer that enables interaction between ground and aerial features.
\end{itemize}

\subsubsection{Token Processing Pipeline}

For each input image pair, the model generates patch embeddings of size $14 \times 14$ pixels, resulting in a $16 \times 16$ grid of 768-dimensional feature vectors. The forward pass can be formulated as:

\begin{align}
\mathbf{F}_g &= \text{DINOv2}(I_g; \mathbf{E}_{pos}^g) \\
\mathbf{F}_a &= \text{DINOv2}(I_a; \mathbf{E}_{pos}^a) \\
\mathbf{F}_{combined} &= \text{Attention}([\mathbf{F}_g; \mathbf{F}_a])
\end{align}

where $\mathbf{F}_g, \mathbf{F}_a \in \mathbb{R}^{16 \times 16 \times 768}$ are the ground and aerial feature matrices, respectively.

\subsection{Orientation-Aware Token Aggregation}

\subsubsection{Sky Filtering and Depth Estimation}

To improve orientation estimation, we incorporate semantic and geometric priors:

\textbf{Sky Segmentation}: We employ a lightweight CNN-based sky filter to identify and mask sky regions in ground images. The sky mask $M_{sky}$ is computed at the patch level using majority voting within each $16 \times 16$ grid cell.

\textbf{Depth Estimation}: We utilize the Depth-Anything model to generate depth maps $D$ for ground images. The depth information is downsampled to match the patch grid, providing depth values $d_{i,j}$ for each spatial location $(i,j)$.

\subsubsection{Multi-Layer Token Aggregation}

We introduce a novel aggregation strategy that separates tokens into three depth layers: foreground, middleground, and background. This approach captures the multi-scale nature of visual features in cross-view matching.

\textbf{Vertical Column Analysis}: For each vertical column $j$ in the ground image feature grid, we compute depth-weighted averages:

\begin{align}
\mathbf{t}_j^{fore} &= \frac{\sum_{i} w_i^{fore} \cdot \mathbf{f}_{i,j}^g \cdot M_{sky}(i,j)}{\sum_{i} w_i^{fore} \cdot M_{sky}(i,j)} \\
\mathbf{t}_j^{mid} &= \frac{\sum_{i} w_i^{mid} \cdot \mathbf{f}_{i,j}^g \cdot M_{sky}(i,j)}{\sum_{i} w_i^{mid} \cdot M_{sky}(i,j)} \\
\mathbf{t}_j^{back} &= \frac{\sum_{i} w_i^{back} \cdot \mathbf{f}_{i,j}^g \cdot M_{sky}(i,j)}{\sum_{i} w_i^{back} \cdot M_{sky}(i,j)}
\end{align}

where the depth-dependent weights are defined as:
\begin{align}
w_i^{fore} &= d_{i,j} \\
w_i^{mid} &= \begin{cases} 
\frac{d_{i,j}}{\tau} & \text{if } d_{i,j} \leq 0.5 \\
\frac{1-d_{i,j}}{d_{i,j}} & \text{otherwise}
\end{cases} \\
w_i^{back} &= 1 - d_{i,j}
\end{align}

with threshold $\tau = 0.5$.

\textbf{Radial Direction Analysis}: For aerial images, we extract features along radial directions from the center:

\begin{align}
\mathbf{r}_\beta^{fore} &= \frac{\sum_{r} w_r^{fore} \cdot \mathbf{f}_{\beta,r}^a}{\sum_{r} w_r^{fore}} \\
\mathbf{r}_\beta^{mid} &= \frac{\sum_{r} w_r^{mid} \cdot \mathbf{f}_{\beta,r}^a}{\sum_{r} w_r^{mid}} \\
\mathbf{r}_\beta^{back} &= \frac{\sum_{r} w_r^{back} \cdot \mathbf{f}_{\beta,r}^a}{\sum_{r} w_r^{back}}
\end{align}

where $\beta$ represents the angle direction, $r$ is the radial distance, and the weights follow a linear progression: $w_r^{fore} = 1-r/R$, $w_r^{mid}$ follows a triangular pattern, and $w_r^{back} = r/R$.

\subsection{Orientation Estimation}

\subsubsection{Cross-Modal Alignment}

We estimate orientation by finding the angular offset that minimizes the cosine distance between corresponding vertical and radial feature aggregations. For each candidate orientation $\theta$, we compute:

\begin{align}
\mathcal{L}(\theta) = \frac{1}{N} \sum_{i=0}^{N-1} \left\| 1 - \begin{bmatrix} \mathbf{t}_{N-1-i}^{fore} \\ \mathbf{t}_{N-1-i}^{mid} \\ \mathbf{t}_{N-1-i}^{back} \end{bmatrix}^T \begin{bmatrix} \mathbf{r}_{\phi(\theta,i)}^{fore} \\ \mathbf{r}_{\phi(\theta,i)}^{mid} \\ \mathbf{r}_{\phi(\theta,i)}^{back} \end{bmatrix} \right\|
\end{align}

where $\phi(\theta,i) = (\lfloor\theta/\Delta\theta\rfloor + i - N/2) \bmod |\mathcal{R}|$ maps vertical columns to radial directions, and $\Delta\theta$ is the angular step size.

\subsubsection{Confidence Estimation}

To assess the reliability of orientation estimates, we compute a confidence score based on the Z-score of the minimum distance:

\begin{align}
\text{confidence} = \frac{\mu(\mathcal{L}) - \min(\mathcal{L})}{\sigma(\mathcal{L})}
\end{align}

where $\mu(\mathcal{L})$ and $\sigma(\mathcal{L})$ are the mean and standard deviation of the loss values across all candidate orientations.

\subsection{Training Strategy}

Our approach operates in a largely unsupervised manner, leveraging the pre-trained DINOv2 features without requiring orientation labels during training. The model learns to align cross-view features through the geometric constraints imposed by the aggregation strategy and the cosine similarity objective.

\subsubsection{Data Preprocessing}

We extract random FOV windows from panoramic images with parameters:
\begin{itemize}
    \item Horizontal FOV: $90^\circ$
    \item Vertical FOV: $180^\circ$ 
    \item Random yaw: $\psi \sim \text{Uniform}(0^\circ, 360^\circ)$
    \item Fixed pitch: $\phi = 90^\circ$
\end{itemize}

Aerial images undergo center cropping and optional polar transformation to align with the ground view geometry.

\subsection{Implementation Details}

The complete pipeline processes image pairs through the following stages:
\begin{enumerate}
    \item Feature extraction using frozen DINOv2-ViT-B/14
    \item Sky segmentation using guided filter refinement
    \item Depth estimation with Depth-Anything model
    \item Multi-layer token aggregation with depth weighting
    \item Cross-modal orientation search with cosine similarity
\end{enumerate}

All models are implemented in PyTorch and can operate on both CPU and GPU. The orientation search space is discretized with angular steps of $\Delta\theta = 90\textdegree/16 = 5.625\textdegree$ for a $16 \times 16$ patch grid.